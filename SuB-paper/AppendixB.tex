%!TEX TS-program = pdflatex
\documentclass[a4paper,12pt]{article}

\usepackage{mathptmx}
\usepackage{parskip}
\usepackage[top=1.5in,left=1in,right=1in,bottom=1in]{geometry}
%\setlength{\parindent}{0pt}
%\setlength{\parskip}{\the\baselineskip}
\pagestyle{empty} 

\makeatletter
\renewcommand\@makefntext[1]{%
  \parindent 1em\noindent
  \hbox{\@makefnmark}#1}
\renewcommand\large{\@setfontsize\large{14pt}{18}} % For standardizing with the Word template: normally \large is 14.4pt
\makeatother
\usepackage{titlesec}
\titleformat{\subsection}{\normalfont\normalsize}{\thesubsection.}{1ex}{}
\titleformat{\subsubsection}{\normalfont\normalsize}{\thesubsubsection.}{1ex}{}
\titleformat{\section}{\normalfont\normalsize\bfseries}{\thesection.}{1ex}{}
\titlespacing*{\section}{0pt}{*0}{0pt}
\titlespacing*{\subsection}{0pt}{\the\baselineskip}{0pt}

\usepackage{natbib} 
\bibpunct{(}{)}{;}{a}{,}{,}

\def\bad{{\leavevmode\llap{*}}}
\def\marginal{{\leavevmode\llap{?}}}
\def\verymarginal{{\leavevmode\llap{??}}}
\def\swmarginal{{\leavevmode\llap{4}}}
\def\infelic{{\leavevmode\llap{\#}}}

\newcommand{\6}{\mbox{$[\hspace*{-.6mm}[$}} 
\newcommand{\9}{\mbox{$]\hspace*{-.6mm}]$}}
\newcommand{\sem}[2]{\6#1\9$^{#2}$}
\newcommand{\mes}[1]{\6#1\9}

\RequirePackage[hyphens]{url}
\usepackage{enumitem,xcolor,color,graphbox,amsmath,amssymb,float,subcaption,hyperref,gb4e}
%\PassOptionsToPackage{hyphens}{url}\usepackage{hyperref}
%\usepackage[hyphens]{url}

\newcommand{\citepos}[1]{\citeauthor{#1}'s \citeyear{#1}}
\newcommand{\citetpos}[1]{\citeauthor{#1}'s (\citeyear{#1})}

\setlength{\bibsep}{0pt} \relax
\setcitestyle{notesep={: },yysep={, }} \relax

\begin{document}

{\large \textbf{On the information structure sensitivity of projective content}}\footnote{We thank Taylor Mahler, Julie McGory and Elena Vaik\v{s}norait\.{e} for assistance in creating the stimuli, as well as the participants of {\em Sinn und Bedeutung} 2018 for critical feedback on the material presented here. We also gratefully acknowledge financial support from National Science Foundation grant BCS-1452674 (JT) and The Ohio State University Targeted Investment in Excellence (MCdM, JT). Jon Stevens was supported by a postdoctoral fellowship awarded to Marie-Catherine de Marneffe and Judith Tonhauser from the American Council of Learned Societies and the OSU Arts \& Sciences Discovery Theme during the year that we designed and ran the experiment.}\\
Judith TONHAUSER --- \textit{The Ohio State University, Stuttgart University}\\
Marie-Catherine DE MARNEFFE --- \textit{The Ohio State University}\\
Shari R.\ SPEER --- \textit{The Ohio State University}\\
Jon STEVENS --- \textit{The Ohio State University}\\

\textbf{Abstract.} This paper presents the findings of an experiment designed to investigate the information structure sensitivity of the projective contents associated with {\em again, stop} and manner adverbs. In contrast to the prejacent of manner adverbs, whose projectivity is expected to be sensitive to information structural focus (e.g., \citealt{abrusan2013,stevens-etal2017}), that of the prejacent of {\em again} is not (e.g., \citealt{beck2006,abrusan2013b}). We also investigated the pre-state content associated with {\em stop} because projection analyses differ in their predictions regarding the sensitivity of this projective content to information structural focus (e.g., \citealt{heim83,vds92,kadmon01,simons01,abrusan2011,abrusan2016,romoli2011,romoli2015}). We found that the projectivity of all three contents is sensitive to prosodically marked focus and discuss the implications of our findings for empirically adequate projection analyses.

\textbf{Keywords:} projective content, presuppositions, {\em stop, again}, manner adverbs, focus, prosody

% deadline Jan 31: send manuscript to sinnub23@gmail.com with "SuB23 Proceedings: FIRST-AUTHOR-LASTNAME" as the subject 
% send TEX, PDF, BIB and all figures
% no strict length restriction, but around 24 pages including everything



\section{Experiment materials: 108 polar questions}\label{a-pqs}

\begin{enumerate}[leftmargin=3ex,itemsep=-4pt]

\item {\bf clean}

\begin{enumerate}[leftmargin=2ex,itemsep=-4pt]
\item Did Annie stop cleaning	/ clean again / clean thoroughly when / before Pete arrived?
\item Did Bennie stop cleaning /clean again / clean thoroughly when / before Rick showed up?
\item Did Charley stop cleaning / clean again / clean thoroughly	when / before Sam came by?
\end{enumerate}

\item {\bf shout}
\begin{enumerate}[leftmargin=2ex,itemsep=-4pt]
\item Did Betty stop shouting / shout again	/ shout wildly when / before the rock show ended?
\item Did Freddy stop shouting / shout again	/ shout wildly	when / before the concert ended?
\item Did Jeanny stop shouting	/ shout again / shout wildly	when / before the protest ended?
\end{enumerate}

\item {\bf yell}
\begin{enumerate}[leftmargin=2ex,itemsep=-4pt]
\item Did Margie stop yelling	/ yell again / yell sternly	when / before the player was injured?
\item Did Nancy stop yelling / yell again / yell sternly when /	before the goalie was hurt?
\item Did Lucy stop yelling	 / yell again / yell sternly when /	before the player was struck?
\end{enumerate}


\item {\bf bark}
\begin{enumerate}[leftmargin=2ex,itemsep=-4pt]

\item Did Fido stop barking	 / bark again / bark aggressively	when / before the storm came?
\item Did Buddy stop barking / bark again / bark aggressively when /	before the mailman came?
\item Did Bella stop barking / bark again / bark aggressively	when / before the lightning struck?
\end{enumerate}


\item {\bf knock}
\begin{enumerate}[leftmargin=2ex,itemsep=-4pt]
\item Did Vinny stop knocking / knock again	 / knock quietly	when / before the bell rang?
\item Did Vicky stop knocking / knock again	 / knock quietly	when / before the phone rang?
\item Did Mickey stop knocking	 / knock again / knock quietly	when / before the alarm rang?
\end{enumerate}

\item {\bf whistle}
\begin{enumerate}[leftmargin=2ex,itemsep=-4pt]
\item  Did Daisy stop whistling	 / whistle again / whistle tunefully when /	before she left?
\item Did Maise stop whistling	/ whistle again	 / whistle tunefully when /	before she went out?
\item Did Rosie stop whistling / whistle again	/ whistle tunefully	when / before she entered?
\end{enumerate}


\item {\bf speak}
\begin{enumerate}[leftmargin=2ex,itemsep=-4pt]
\item Did Maddy stop speaking / speak again	/ speak casually when /	before her boss entered?
\item Did Zadie stop speaking	/ speak again	/ speak casually when / before the director came in?
\item Did Johnny stop speaking / speak again	/ speak casually when / before his super arrived?
\end{enumerate}


\item {\bf walk}
\begin{enumerate}[leftmargin=2ex,itemsep=-4pt]
\item Did Lenny stop walking	/ walk again / walk slowly	when / before the sun went down?
\item Did Molly stop walking / walk again / walk slowly	when / before the rain began?
\item Did Abby stop walking	/  walk again	/ walk slowly when / before the storm started?
\end{enumerate}



\item {\bf smile}
\begin{enumerate}[leftmargin=2ex,itemsep=-4pt]
\item Did Eddie stop smiling / smile again / smile ironically when /	before the tune played?
\item Did Teddy stop smiling / smile again / smile ironically	when / before the song came on?
\item Did Danny stop smiling / smile again	/ smile ironically	when / before the band played?
\end{enumerate}


\item {\bf cry}
\begin{enumerate}[leftmargin=2ex,itemsep=-4pt]
\item Did Jamie stop crying	/ cry again	/ cry loudly when /	before the verdict came in?
\item Did Addy stop crying	/ cry again	/ cry loudly when / before the award was announced?
\item Did Katie stop crying	/ cry again	/ cry loudly when / before the prize was revealed?
\end{enumerate}


\item {\bf text} 
\begin{enumerate}[leftmargin=2ex,itemsep=-4pt]
\item Did Robbie stop texting	/ text again	/ text secretly	when / before Kim showed up?
\item Did Sophie stop texting / text again	/ text secretly	when / before Tom walked in?
\item Did Alfie stop texting  / text again	/ text secretly	when / before Dan arrived?
\end{enumerate}


\item {\bf giggle}
\begin{enumerate}[leftmargin=2ex,itemsep=-4pt]
\item Did Harry stop giggling / giggle again / giggle nervously	when / before the interview started?
\item Did Ronny stop giggling	/ giggle again	/ giggle nervously	when / before the meeting began?
\item Did Henry stop giggling / giggle again / giggle nervously	when / before the exam started?

\end{enumerate}
\end{enumerate}

\section{Details on the creation of the prosodic stimuli}\label{a-splicing}

{\bf Who produced the target stimuli? Where? Which recording equipment?}

\subsection{Cutting the recordings}

\begin{itemize}

\item Specifically, we cut each recording of the 432 utterances of the 108 polar questions at the onset of the first vowel after {\em when} or {\em before} in the temporal adjunct clause; e.g., at the onset of the [i] of {\em Pete} in the sample stimuli in (\ref{certain}).

\item We then created four new recordings of utterances of the polar question in the four prosodic conditions by splicing the four recordings of the utterances before the cut point (i.e., for the polar questions in (\ref{certain}), the recording of {\em Did Annie stop cleaning when/before P}) with the recording of the one of the adjunct clauses. As a consequence, the recordings of the four utterances of each polar question do not differ in the material after the cut point.

\item pick the four utterances with the best main clauses (PA on relevant words, no funny business)

\item 0-crossing, cut after first vowel after before/when

\end{itemize}

\subsection{Splicing and selecting utterances for stimuli}

\begin{itemize}

\item PROCEDURE FOR PICKING ADJUNCT TO SPLICE

if intra-condition subj adjunct splice is bad (e.g., too much of an f0 change, no rise at the end of adjunct clause):

-if it's a "when" condition (stop/again), try the other subj "when", then try subj "before"

-if it's a "before" condition, try subj "when" from again, then subj "when" from stop

-if the above fails:

--go to third pitch accent (verb for adverb and again, target for stop), repeat same procedure

\item which adjunct clauses did we ultimately use?

\end{itemize}

\subsection{Norming studies}\label{a-norming}

To ensure that the prosodic stimuli had the desired qualities, we conducted two norming studies. The first study checked whether the location of the pitch accent in the main clause was clearly identifiable. We randomly selected a subset of 44 stimuli, i.e., 10\% of the 432 stimuli, with 11 stimuli from each of the four prosodic conditions. A native speaker of English listened to the 44 target stimuli and correctly identified the location of the pitch accent in all of them. The second study checked whether participants would be able to identify that some of the stimuli were utterances of sentences with manner adverbs. A non-native speaker of English listened to 36 utterances with manner adverbs (the subset of the manner adverb utterances that was produced with a pitch accent on the auxiliary). She correctly identified the 12 manner adverbs in the 36 stimuli. However, because of comments we received during the norming studies, we replaced, for the verb {\em whistle}, the manner adverb {\em tunefully} with the manner adverb {\em beautifully} in the final set of target stimuli.




\bibliographystyle{chicago}
\bibliography{/Users/tonhauser.1/Documents/bibliography}

\end{document}
